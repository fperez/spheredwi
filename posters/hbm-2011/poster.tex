%\documentclass[landscape,a0b,final,a4resizeable]{a0poster}
\documentclass[landscape,a0b,final]{a0poster}
%\documentclass[portrait,a0b,final,a4resizeable]{a0poster}
%\documentclass[portrait,a0b,final]{a0poster}
%%% Option "a4resizeable" makes it possible ot resize the
%   poster by the command: psresize -pa4 poster.ps poster-a4.ps
%   For final printing, please remove option "a4resizeable" !!


\usepackage{amsthm}
\usepackage{amsbsy}
\usepackage{amssymb}
\usepackage{epsfig}
\usepackage{multicol}
\usepackage{pstricks,pst-grad}



%%%%%%%%%%%%%%%%%%%%%%%%%%%%%%%%%%%%%%%%%%%
% Definition of some variables and colors
%\renewcommand{\rho}{\varrho}
%\renewcommand{\phi}{\varphi}
\setlength{\columnsep}{3cm}
\setlength{\columnseprule}{2mm}
\setlength{\parindent}{0.0cm}

\newtheorem{thm}{Theorem}

\newcommand{\Har}{{\mathcal H}}

\newcommand{\HH}{{\mathcal P}}

\newcommand{\Qu}{{\mathbb H}}

\newcommand{\bomega}{{\boldsymbol{\omega}}}

\newcommand{\odotop}{{\boldsymbol{\omega}\cdot\boldsymbol{\omega}'}}


%%%%%%%%%%%%%%%%%%%%%%%%%%%%%%%%%%%%%%%%%%%%%%%%%%%%
%%%               Background                     %%%
%%%%%%%%%%%%%%%%%%%%%%%%%%%%%%%%%%%%%%%%%%%%%%%%%%%%

\newcommand{\background}[3]{
  \newrgbcolor{cgradbegin}{#1}
  \newrgbcolor{cgradend}{#2}
  \psframe[fillstyle=gradient,gradend=cgradend,
  gradbegin=cgradbegin,gradmidpoint=#3](0.,0.)(1.\textwidth,-1.\textheight)
}



%%%%%%%%%%%%%%%%%%%%%%%%%%%%%%%%%%%%%%%%%%%%%%%%%%%%
%%%                Poster                        %%%
%%%%%%%%%%%%%%%%%%%%%%%%%%%%%%%%%%%%%%%%%%%%%%%%%%%%

\newenvironment{poster}{
  \begin{center}
  \begin{minipage}[c]{0.98\textwidth}
}{
  \end{minipage} 
  \end{center}
}



%%%%%%%%%%%%%%%%%%%%%%%%%%%%%%%%%%%%%%%%%%%%%%%%%%%%
%%%                pcolumn                       %%%
%%%%%%%%%%%%%%%%%%%%%%%%%%%%%%%%%%%%%%%%%%%%%%%%%%%%

\newenvironment{pcolumn}[1]{
  \begin{minipage}{#1\textwidth}
  \begin{center}
}{
  \end{center}
  \end{minipage}
}



%%%%%%%%%%%%%%%%%%%%%%%%%%%%%%%%%%%%%%%%%%%%%%%%%%%%
%%%                pbox                          %%%
%%%%%%%%%%%%%%%%%%%%%%%%%%%%%%%%%%%%%%%%%%%%%%%%%%%%

\newrgbcolor{lcolor}{0. 0. 0.80}
\newrgbcolor{gcolor1}{1. 1. 1.}
\newrgbcolor{gcolor2}{.80 .80 1.}

\newcommand{\pbox}[4]{
\psshadowbox[#3]{
\begin{minipage}[t][#2][t]{#1}
#4
\end{minipage}
}}



%%%%%%%%%%%%%%%%%%%%%%%%%%%%%%%%%%%%%%%%%%%%%%%%%%%%
%%%                myfig                         %%%
%%%%%%%%%%%%%%%%%%%%%%%%%%%%%%%%%%%%%%%%%%%%%%%%%%%%
% \myfig - replacement for \figure
% necessary, since in multicol-environment 
% \figure won't work

\newcommand{\myfig}[3][0]{
\begin{center}
  \vspace{1.25cm}
  \includegraphics[width=#3\hsize,angle=#1]{#2}
  \nobreak\medskip
\end{center}}



%%%%%%%%%%%%%%%%%%%%%%%%%%%%%%%%%%%%%%%%%%%%%%%%%%%%
%%%                mycaption                     %%%
%%%%%%%%%%%%%%%%%%%%%%%%%%%%%%%%%%%%%%%%%%%%%%%%%%%%
% \mycaption - replacement for \caption
% necessary, since in multicol-environment \figure and
% therefore \caption won't work

%\newcounter{figure}
\setcounter{figure}{1}
\newcommand{\mycaption}[1]{
  \vspace{0.25cm}
  \begin{quote}
    {{\sc Figure} \arabic{figure}: #1}
  \end{quote}
  \vspace{0.25cm}
  \stepcounter{figure}
}



%%%%%%%%%%%%%%%%%%%%%%%%%%%%%%%%%%%%%%%%%%%%%%%%%%%%%%%%%%%%%%%%%%%%%%
%%% Begin of Document
%%%%%%%%%%%%%%%%%%%%%%%%%%%%%%%%%%%%%%%%%%%%%%%%%%%%%%%%%%%%%%%%%%%%%%

\begin{document}

\background{1. 1. 1.}{1. 1. 1.}{0.5}

\vspace*{2cm}


\newrgbcolor{lightblue}{0. 0. 0.80}
\newrgbcolor{white}{1. 1. 1.}
\newrgbcolor{whiteblue}{.80 .80 1.}


\begin{poster}

%%%%%%%%%%%%%%%%%%%%%
%%% Header
%%%%%%%%%%%%%%%%%%%%%
\begin{center}
\begin{pcolumn}{0.98}

\pbox{0.95\textwidth}{}{linewidth=2mm,framearc=0.3,linecolor=lightblue,fillstyle=gradient,gradangle=0,gradbegin=white,gradend=whiteblue,gradmidpoint=1.0,framesep=1em}{

%%% Pic1
\begin{minipage}[c][9cm][c]{0.1\textwidth}
  \begin{flushleft}
    \myfig[0]{UCBerkeleyLogo.eps}{0.9}
  \end{flushleft}
\end{minipage}
%%% Title
\begin{minipage}[c][9cm][c]{0.78\textwidth}
  \begin{center}
    {\sc \Huge Alternate grids for diffusion weighted imaging \\ [7mm] and associated
reconstruction algorithms}\\ [5mm]
    {\Large Fernando Perez and St\'{e}fan van der Walt\\
    University of California Berkeley, Berkeley CA \\[5mm]
    Cory Ahrens\\
    Colorado School of Mines, Golden CO}
  \end{center}
\end{minipage}
%%% Pic2
\begin{minipage}[c][9cm][c]{0.1\textwidth}
  \begin{flushleft}
    \myfig[0]{CSM_Logo.eps}{0.9}
  \end{flushleft}
\end{minipage}

}
\end{pcolumn}
\end{center}


\vspace*{1cm}


%%% Begin of Multicols-Enviroment
\begin{multicols}{3}


%%% Introduction
\begin{center}\pbox{0.8\columnwidth}{}{linewidth=2mm,framearc=0.1,linecolor=lightblue,fillstyle=gradient,gradangle=0,gradbegin=white,gradend=whiteblue,gradmidpoint=1.0,framesep=1em}{\begin{center}Introduction\end{center}}\end{center}
\vspace{0.25cm}
Many problems in physics, mathematics and engineering involve integration
and interpolation on the sphere in $\mathbb{R}^{3}$. Of particular
importance are discretizations of rotationally invariant subspaces
of $L^{2}\left(\mathbb{S}^{2}\right)$ that integrate all spherical
harmonics up to a fixed order and degree. Typically, the sphere is discretized using
equally spaced nodes in azimuthal angle and Gauss-Legendre nodes in polar angle,
leading to an unreasonably dense concentration of nodes near the poles. In a variety
of applications such concentration of nodes may lead to problems when using
these grids. Alternatively, Sobolev \cite{SOBOLEV-1962} suggested the use of
grids that are invariant under finite rotation groups. In such constructions
there is no clustering of nodes and, moreover, the number of nodes necessary to
integrate a particular subspace is close to optimal. 

We develop a systematic numerical approach for constructing
nearly optimal quadratures invariant under the icosahedral group to
integrate rotationally invariant subspaces of $L^{2}\left(\mathbb{S}^{2}\right)$
up to a fixed order and degree. Using these grids and a reproducing
kernel, we show how to replace the standard basis of spherical harmonics
by a representation formed using a single function centered at the
quadrature nodes. The reproducing kernel is mostly concentrated near 
the corresponding grid point. In this representation, the 
coefficients, up to a factor, are the values on the grid of the function 
being represented. We may interpret this construction as an analogue of 
Lagrange interpolation on the sphere. 

We view our approach as the first step in constructing a local and
multiresolution representation of functions on the sphere that respects
rotationally invariant subspaces.  


%%% Preliminaries
\vspace{0.25cm}
\begin{center}\pbox{0.8\columnwidth}{}{linewidth=2mm,framearc=0.1,linecolor=lightblue,fillstyle=gradient,gradangle=0,gradbegin=white,gradend=whiteblue,gradmidpoint=1.0,framesep=1em}{\begin{center}Preliminaries\end{center}}\end{center}
\vspace{0.25cm}

Here we establish notation and state some well known results about
spherical harmonics, reproducing kernels and state two theorems by Sobolev.

We denote the unit sphere in $\mathbb{R}^{3}$ as $\mathbb{S}^{2}$. An orthonormal basis for $L^{2}\left(\mathbb{S}^{2}\right)$ is given by the spherical harmonics, 
\begin{equation}
Y_{n}^{m}\left(\theta,\phi\right)=\frac{1}{\sqrt{2\pi}}\overline{P}_{n}^{m}\left(\cos\theta\right)e^{im\phi},\;0\le|m|\le n,\quad n=0,1,\dots,\label{eq:ylm}
\end{equation}
where the polar angle $\theta\in[0,\pi]$, the azimuthal angle $\phi\in[0,2\pi)$
and $\overline{P}_{n}^{m}$ are the normalized associated Legendre
functions. We define a subspace of spherical harmonics with fixed degree $n$ as
\begin{equation}
\Har_{n}=\mathrm{span}\left\{ Y_{n}^{m}\left(\theta,\phi\right),\,\:|m|\le n\right\} .\label{eq:space-sp-h-fixed-degree}
\end{equation}
The dimension of $\Har_{n}$ is $2n+1$. The subspace of maximum
degree $N$ is then the direct sum
\begin{equation}
\HH_{N}=\bigoplus_{n=0}^{N}\Har_{n}=\mathrm{span}\left\{ Y_{n}^{m}\left(\theta,\phi\right),\:|m|\le n,\:0\le n\le N\right\} \label{eq:fsubsp}
\end{equation}
and has dimension $\left(N+1\right)^{2}$. The reproducing kernel for $\HH_{N}$, 
\begin{equation}
K\left(\bomega\cdot\bomega'\right) = \sum_{n=0}^{N}\frac{2n+1}{4\pi}P_{n}\left(\odotop\right),\label{eq:kernel}
\end{equation}
satisfies 
\begin{equation}
f\left(\bomega\right)=\int_{\mathbb{S}^{2}}K\left(\odotop\right)f\left(\bomega'\right)d\Omega',\,\,\,\, f\in\HH_{N}.\label{eq:rep}
\end{equation}
The identity in (\ref{eq:rep}) may be verified by using the addition theorem for spherical harmonics.
We rely on (\ref{eq:rep}) to develop a representation of functions
in $\HH_{N}$ which is an analogue of Lagrange interpolation on the sphere, known as hyper-interpolation.

Sobolev's paper \cite{SOBOLEV-1962} contains two key results that
we now summarize, specialized to the icosahedral rotation group:
\begin{thm}
\label{thm:invQ}Let Q be a quadrature rule invariant under the group
G. Then Q is exact for all functions $f\in\HH_{N}$ if and only if
Q is exact for functions $f$ invariant under G. 
\end{thm}
This theorem reduces the size of the system of nonlinear
equations which must be solved to determine a quadrature invariant
under the group $G$. The next result gives a formula to calculate
the number of invariant functions under the group $G$ in a subspace
of spherical harmonics $\Har_{n}$ of a given degree $n$. Let $q_{1}=5$
be the number of edges meeting at a vertex of an icosahedron, $q_{2}=3$
be the number of sides of its (triangular) face and $q_{3}=2$ denote
the order of rotation about mid-points of opposing edges. 
\begin{thm}
\label{thm:Snumbers}For a given degree $n$, the number of functions
invariant under the icosahedral rotation group in a subspace of spherical
harmonics $\Har_{n}$ is given by
\[
S(n)=\left\lfloor \frac{n}{q_{1}}\right\rfloor +\left\lfloor \frac{n}{q_{2}}\right\rfloor +\left\lfloor \frac{n}{q_{3}}\right\rfloor -n+1,
\]
where $\left\lfloor \,\right\rfloor$ denotes the integer part.
\end{thm}

%%% Quadratures
\vspace{0.25cm}
\begin{center}\pbox{0.8\columnwidth}{}{linewidth=2mm,framearc=0.1,linecolor=lightblue,fillstyle=gradient,gradangle=0,gradbegin=white,gradend=whiteblue,gradmidpoint=1.0,framesep=1em}{\begin{center}Quadratures for the sphere\end{center}}\end{center}
\vspace{0.25cm}

The main difficulty in constructing quadratures comes from the need
to solve a large system of nonlinear equations. Without using special
structure of these equations, general root finding or optimization
methods typically fail. The essence of our approach is to develop
and use such structure within a root finding method.

To start, there are four different types of orbits of the icosahedral
rotation group. In general, a point on the sphere under the action
of the group generates a total of $60$ points. However, if a point
is a vertex of the icosahedron, then it generates a total of only
$12$ distinct points. Also, if a point is the projection of the center
of an icosahedron face onto the sphere, it generates $20$ distinct
points in total. Finally, if a point is the projection onto the sphere
of the mid-point of an edge, it generates a total of $30$ distinct
points. When describing these different types of orbits it is sufficient
to consider a single point, a generator of its orbit. The orbit of
a point with spherical coordinates $\left(\theta,\phi\right)$ is
the set $\left\{ \left(g_{i}^{-1}\theta,g_{i}^{-1}\phi\right)\:|\: g_{i}\in G\right\} $
and, depending on the type of orbit, has cardinality $12$, $20$,
$30$ or $60$.

With these types of orbits in mind, we consider four types of quadratures.
The first type assumes that all generators, except for a vertex of
the icosahedron, give rise to orbits of size $60$, i.e., \begin{equation}
Q_{v}\left(f\right)=w_{v}\sum_{i=1}^{12}f\left(\theta_{i}^{v},\phi_{i}^{v}\right)+\sum_{j=1}^{N_{g}}w_{j}\sum_{i=1}^{60}f\left(\theta_{i}^{\left(j\right)},\phi_{i}^{\left(j\right)}\right),\label{eq:Q-v}\end{equation}
where $\left\{ \theta_{i}^{v},\phi_{i}^{v}\right\} _{i=1}^{12}$ are
coordinates of the vertices of an icosahedron inscribed in the unit
sphere, $w_{v}$ their associated weight, $N_{g}$ is the number of
generators with coordinates $\left\{ \theta^{\left(j\right)},\phi^{\left(j\right)}\right\} _{j=1}^{N_{g}}$
and weights $\left\{ w_{j}\right\} _{j=1}^{N_{g}}$. For each $g_{i}\in G$,
we denote $\left(\theta_{i}^{\left(j\right)},\phi_{i}^{\left(j\right)}\right)=\left(g_{i}^{-1}\theta^{\left(j\right)},g_{i}^{-1}\phi^{\left(j\right)}\right)$.

The second type of quadrature has the form
\begin{equation}
Q_{vf}\left(f\right)=w_{v}\sum_{i=1}^{12}f\left(\theta_{i}^{v},\phi_{i}^{v}\right)+w_{f}\sum_{i=1}^{20}f\left(\theta_{i}^{f},\phi_{i}^{f}\right)+\sum_{j=1}^{N_{g}}w_{j}\sum_{i=1}^{60}f\left(\theta_{i}^{\left(j\right)},\phi_{i}^{\left(j\right)}\right),\label{eq:Q-vf}
\end{equation}
where $\left\{ \theta_{i}^{f},\phi_{i}^{f}\right\} _{i=1}^{20}$ are
coordinates of the face centers of the icosahedron projected onto
the sphere and $w_{f}$ is the associated weight.

The other quadratures are constructed in a similar manner. Using Theorem~\ref{thm:Snumbers}, we determine the number of invariant functions in the subspace $\Har_{n}$.Theorem~\ref{thm:invQ} allows us to limit the number of equations to not exceed the number of invariant functions. We solve the resulting system of equations using Newtons's method.

Figure 1 shows a type 1 quadrature for $\HH_{145}$ with 7121 nodes.
%\begin{center}
  % first argument: eps-file
  % second argument: stretching-factor relative to Column-width (<1)
  % optional argument: rotation angle (0-360), default=0
%  \myfig[0]{N145.eps}{0.25}
%  \mycaption{Type 1 quadrature for degree $N=145$.}
%\end{center}
%
%\begin{center}
%  \includegraphics[scale=0.45]{N145.eps}
%  %\vspace{0.25cm}
%  \begin{quote}{Figure 1: Type 1 quadrature for degree $N=145$.}\end{quote}
%\end{center}

%\vspace{3cm}



%
A useful way of measuring the efficiency of a quadrature is the ratio of the dimension of the subspace
to be integrated to the (maximum) number of degrees of freedom in
the quadrature rule (two coordinates and a weight for each node):
%
\begin{equation}
\eta=\frac{\left(N+1\right)^{2}}{3M},\label{eq:efficiency}
\end{equation}
%
where $M$ is the number of nodes. If $\eta=1$, we call the quadrature optimal.
%
%\begin{center}
%  \includegraphics[scale=1.5]{Eff_new.eps}
%  \vspace{0.25cm}
%  \begin{quote}{Figure 2: Potential efficiency (\ref{eq:efficiency})
%of quadrature (\ref{eq:Q-v}) as a function of degree $N$ of subspace
%$\HH_{N}$ computed using Theorem~\ref{thm:Snumbers}. Also shown is the efficiency of the standard 
%quadrature $\eta=2/3$.}\end{quote}
%\end{center}
%
In Figure~2 we display the
potential efficiency of quadrature $Q_{v}$ as a function of the degree
$N$ of subspace $\HH_{N}$ using Theorem~\ref{thm:Snumbers} to count invariants and superimpose the
actual efficiencies of computed quadratures. The behavior of efficiency
of other quadratures is similar. For comparison, efficiency of the standard quadrature $\eta=2/3$
is also shown. Solid dots indicate the efficiency of computed quadratures
using our approach.



%%%Local representations
\vspace{0.25cm}
\begin{center}\pbox{0.8\columnwidth}{}{linewidth=2mm,framearc=0.1,linecolor=lightblue,fillstyle=gradient,gradangle=0,gradbegin=white,gradend=whiteblue,gradmidpoint=1.0,framesep=1em}{\begin{center}Local representations\end{center}}\end{center}
\vspace{0.25cm}

Using the new quadratures, we construct an alternative representation for functions on invariant
subspaces of $L^{2}\left(\mathbb{S}^{2}\right)$ by discretizing Eq. (\ref{eq:rep})
\begin{equation}
f\left(\boldsymbol{\omega}\right)=\sum_{j=1}^{M}K\left(\bomega\cdot\bomega_{j}\right)w_{j}f\left(\bomega_{j}\right).\label{eq:f-semi-disc}
\end{equation}
If $f\in\HH_{N}$, then (\ref{eq:f-semi-disc}) provides
an exact reconstruction of the function $f$ from its values $f\left(\bomega_{1}\right),f\left(\bomega_{2}\right),\dots,f\left(\bomega_{M}\right)$.
The functions $\left\{ K\left(\bomega\cdot\bomega_{j}\right)w_{j}\right\} _{j=1}^{M}$
play a role similar to that of Lagrange interpolating polynomials
and, therefore, we may think of (\ref{eq:f-semi-disc}) as an analogue
of Lagrange interpolation on the sphere.

Further localization of $K$ may be achieved by optimization techniques.



%%% This starts another column
%%% Section
\vspace{0.25cm}
\begin{center}\pbox{0.8\columnwidth}{}{linewidth=2mm,framearc=0.1,linecolor=lightblue,fillstyle=gradient,gradangle=0,gradbegin=white,gradend=whiteblue,gradmidpoint=1.0,framesep=1em}{\begin{center}Conclusions\end{center}}\end{center}
\vspace{0.25cm}

We introduced a numerical method for constructing quadratures
invariant under the icosahedral group which integrate rotationally
invariant subspaces of $L^{2}\left(\mathbb{S}^{2}\right)$. Using these quadratures,
an exact representation of functions on rotationally invariant
subspaces of $L^{2}\left(\mathbb{S}^{2}\right)$, similar to Lagrange
interpolation, was developed. The results of this paper are the first step to develop practical computational methods for applications that deal with the sphere. See \cite{AHR-BEY-2009} for more details.

\vspace{0.25cm}
\emph{This research was partially supported by AFOSR grant FA9550-07-1-0135,
NSF grant DMS-0612358, DOE/ORNL grants 4000038129 and DE-FG02-03ER25583.}


%%% References
\bibliographystyle{plain}
\bibliography{poster}


\end{multicols}

\end{poster}

\end{document}

