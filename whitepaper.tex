\documentclass[10pt]{article}
\usepackage{fancyhdr,amssymb,amsmath,graphicx}
\usepackage{hyperref,fullpage,mdwlist}
\setlength{\topmargin}{-36pt}
\setlength{\headheight}{42pt}
\setlength{\headsep}{5pt}
\setlength{\footskip}{20pt}
\renewcommand{\labelitemi}{{\Large\textperiodcentered}}
\pagenumbering{empty}
\pagestyle{fancy}
% ----------------------------------------

\begin{document}
%\layout
\sloppy
\lhead{{\Huge Tech-X \\ Corporation}}
\rhead{{\large 5621 Arapahoe Ave, Suite A\\
       \large Boulder, CO 80303}}
\chead{\parbox{5cm}{}\includegraphics[width=65pt]{TX_Logo_rev.pdf}}

%Second thicker line
\setlength{\unitlength}{1in}
\begin{picture}(5,0.025)
  \linethickness{0.5mm}
  \put(-0.2,-0.074){\line(1,0){6.5}}
\end{picture}

%Footer infomation
\lfoot{{Phone: 303.448.0727}}
\cfoot{{\url{www.txcorp.com}}}
\rfoot{{Fax: 303.448.7756}}

%Title
\vspace{-2pt}
\begin{center} {\huge{
      New Computationally Efficient  \\
      \vspace{3pt} Algorithms for Diffusion MRI} }
\end{center}

\vspace{-5pt}
\section*{Summary}
\vspace{-5pt}
Diffusion weighted MRI (dMRI) techniques hold the promise of \emph{in vivo} imaging of
fiber tracts in the brain. This type of imaging is proving to be of enormous
value in understanding the connection structure of the brain, an essential
component for description of both healthy and diseased brain function.

We propose to develop an open source library for analysis, diagnosis and
visualization of dMRI datasets that includes novel algorithmic
approaches to representing the underlying data and optimized
components based on use of specialized fast hardware available today.  This
library will be contributed to existing NIH-funded open source tools for
neuroimaging and will form the basis for developing an FDA-approved,
commercially supported package for clinical analysis of dMRI data.

The analysis of dMRI data requires estimating the direction of
motion of water molecules at each location (voxel) in the brain, providing an
Orientation Distribution Function (ODF). Standard approaches represent
the ODF using expansions in spherical harmonics, the spherical analog of
sines and cosines in classical Fourier analysis. Like sines and cosines, spherical harmonics
are unable to efficiently and accurately represent localized features. For dMRI datasets, this is
of particular concern, since many key features of the data arise precisely from complex local patterns of anatomy
such as crossing neural fibers.

Wavelet techniques are a natural mathematical tool for analysis of signals with
localized features and they have been successfully used in many fields. However, there
are fundamental mathematical difficulties in constructing wavelets for spherical
domains, and only recently have these been successfully tackled, with the
introduction of ``needlets'', a wavelet-like construction that respects the
topology of the sphere. When using either needlets or spherical harmonics, however, the 
processing of dMRI data is computationally intensive and time consuming.

Use of graphical processing units (GPUs) in scientific computing has
increased tremendously. GPUs offer massively-parallel processing capabilities
and it is not uncommon to achieve a factor of 50 in speed up over CPU-based
codes. Because of the independence of voxels and the amount of data in
dMRI, use of GPUs will provide significant performance improvements.

By combining the expertise in computational harmonic analysis and GPU computing
at Tech-X corporation with the expertise in brain imaging and open source software
development at the Henry H. Wheeler Brain Imaging Center at UC Berkeley, we propose to
\begin{itemize*}
\item{Address the problem of robustly finding localized features in dMRI data by developing new
      needlet based algorithms.}
\item{Address the computational burden of processing dMRI data by developing software
     libraries implementing needlet algorithms which take advantage of massively parallel GPU architectures.}
\item{Address developing a commercial software package by first releasing open-source software that is open to scrutiny
      of the scientific community and then build commercial packages from open-source code.}
\end{itemize*}
The result of this collaboration will be to provide the research and clinical
medical communities with unique, state-of-the-art dMRI software.

\vspace{-8pt}
\section*{Proposed technical approach}
\vspace{-5pt}
The reconstruction of dMRI data can be cast as a
deconvolution \cite{Lenglet2008}. The problem is
to reconstruct the probability of diffusion $p({\mathbf r},\tau)$, for each
voxel, from the equation $S\left({\mathbf q},\tau\right) = \mathcal{F}_{\mathbf r}\left[p({\mathbf r},\tau)\right]$ where $S$ is the signal attenuation measured when a field gradient in
direction ${\mathbf q}$ acts over a time interval $\tau$ and $\mathcal{F}_{\mathbf r}\left[\cdot\right]$ is
the 3D Fourier transform.  The full probability $p$ can be approximated by sampling
with field gradients over the surface of a sphere at fixed gradient strength, which provides the 
ODF. We propose to estimate the ODF using ideas from
\cite{KE-NG-PI-2009} with needlets constructed using new quadratures from
\cite{AHR-BEY-2009} and optimized filters.  The ability of needlets to
faithfully represent localized features will provide more stable
tractography analysis, as the artifacts inherent in global bases
(e.g. spherical harmonics) can lead to spurious fiber identifications and other
problems.  In order to accurately estimate higher order coefficients of the
needlet expansion, the ODF reconstruction will be cast as a sparse reconstruction
problem, leading to many $(\sim 10^6)$ small $(\lesssim 256\times256)$ independent linear
systems of equations (one per voxel) which can be efficiently solved on GPUs. Tech-X currently
has libraries of GPU-accelerated numerical solvers for tackling these problems.

\vspace{-8pt}
\section*{Commercialization}
\vspace{-5pt}
Tech-X was founded in Boulder, CO in 1994. With a staff of more than 60 people,
of whom 60\% have Ph.D. degrees in physics, mathematics, engineering, and
computer science, Tech-X has strong expertise in computational simulations, GPU
software development, data analysis, and distributed computing applied to
engineering and scientific applications.

In the initial phase of this project, we will develop an open source library
that will be incorporated into the Nipy project (http://nipy.sourceforge.net),
an NIH-funded collection of open source tools for analysis of neuroimaging data.
All software at this stage will be contributed to Nipy under the terms of the
BSD Software License, which allows for further commercial development. By developing
open source software, the broader scientific community will be to analyze and test
our new methodologies. This will allow for continuous community critique and 
development, which ultimately will define the utility of this approach. Once the
methodologies gain traction in the neuroimaging community, Tech-X will then develop
an FDA approved, commercial software package for clinical use. During phase I, we will
explore the current market conditions for this technology and develop a commercialization
plan based on those results.

\vspace{-8pt}
\section*{Key personnel}
\vspace{-5pt}
This project brings together of number of individuals with varying expertise to make this project successful. These individuals are:
\begin{itemize*}
  \item{Dr. Fernando Perez received his PhD in Theoretical Physics from the University of Colorado at Boulder in 2002. His area of expertise is in development of modern open source tools for scientific computing in Python and is an Associate Researcher at the Brain Imaging Center of the UC, Berkeley.}

  \item{Dr. Matthew Brett trained as a neurologist before transferring to full-time research in 1996. His area of expertise is in imaging methodologies in positron emission and functional and diffusion MRI.}
  
  \item{Dr. Paul Mullowney received his PhD in Applied Mathematics at the University of Colorado at Boulder. His area of expertise is in high-performance computing, specifically with GPUs, and is a research mathematician at Tech-X Corp.}

  \item{Dr. Cory Ahrens received Ph.D. in Applied Mathematics at the University of Colorado at Boulder in 2006. His area of expertise is in developing wavelet based algorithms for the sphere, and is an associate research mathematician at Tech-X Corp.}
  
\end{itemize*}

\bibliographystyle{plain}
\bibliography{biblio,diffusion_needlets}

\end{document}
