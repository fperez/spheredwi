\documentclass[10pt]{article}
\usepackage{fancyhdr,amssymb,amsmath,graphicx}
\usepackage{hyperref,fullpage}
\setlength{\topmargin}{-36pt}
\setlength{\headheight}{42pt}
\setlength{\headsep}{10pt}
\setlength{\footskip}{0pt}
\renewcommand{\labelitemi}{{\Large\textperiodcentered}}
\pagenumbering{empty}
\pagestyle{fancy}
% ----------------------------------------

\begin{document}
%\layout
\sloppy
\lhead{{\Huge Tech-X \\ Corporation}}
\rhead{{\large 5621 Arapahoe Ave, Suite A\\
       \large Boulder, CO 80303}}
\chead{\parbox{5cm}{}\includegraphics[width=65pt]{TX_Logo_rev.pdf}}

%Second thicker line
\setlength{\unitlength}{1in}
\begin{picture}(5,0.025)
  \linethickness{0.5mm}
  \put(-0.2,-0.074){\line(1,0){6.5}}
\end{picture}

%Footer infomation
\lfoot{{Phone: 303.448.0727}}
\cfoot{{\url{www.txcorp.com}}}
\rfoot{{Fax: 303.448.7756}}

\newcommand{\bomega}{{\boldsymbol{\omega}}}

%Title
\vspace{5pt}
\begin{center} {\huge{
      New computationally efficient  \\
      \vspace{10pt} algorithms for Diffusion MRI} }
\end{center}

\section*{Summary}

We propose to develop an open source library for the analysis, diagnosis and
visualization of diffusion imaging datasets that includes novel algorithmic
approaches to the representation of the underlying data as well as optimized
components based on the use of specialized fast hardware available today.  This
library, contributed to existing NIH-funded open source tools for neuroimaging,
will form the basis for the development of an FDA-approved, commercially supported 
package for clinical analysis of diffusion imaging data.

Diffusion weighted MRI techniques hold the promise of \emph{in vivo} imaging of fiber
tracts in the brain; this is proving to be of enormous value in understanding the
connection structure of the brain, an essential component for the description
of both healthy and diseased brain function.  

The analysis of diffusion weighted images requires estimating the direction of
motion of water molecules at each location (voxel) in the brain, providing an
Orientation Distribution Function (ODF). Standard analytic approaches to this
problem represent the ODF using expansions in Spherical Harmonics, a set of
functions that constitute the spherical analog of sines and cosines in
classical Fourier analysis.  Spherical Harmonics share some key limitations of
sines and cosines, namely their inability to efficiently and accurately
represent localized features in signals.  For the study of diffusion data, this
is of particular concern, as many of the key features of the data arise
precisely from complex local patterns of anatomy such as crossing neural
fibers.

Wavelet techniques are a natural mathematical tool for the analysis of signals
with localized features, and they have been successfully used in many fields
where the signals can be defined on domains whose structure is that of a line,
plane or cube.  However, there are fundamental mathematical difficulties in
constructing the analog of wavelets for a spherical domain, and only recently have
these been successfully tackled, with the introduction of
``needlets'', a wavelet-like construction that respects the specific
topological constraints of working on the surface of a sphere.

The use of graphical processing units (GPUs) in scientific computing has
increased tremendously. GPUs offer massively-parallel processing capabilities
and it is not uncommon to achieve a factor of 50 in speed up over CPU-based
codes. Because of the independence of voxels and the amount of data in
diffusion MRI, use of GPUs to process diffusion MRI data will offer significant
performance improvements.

By combining the expertise in computational harmonic analysis and GPU computing
at Tech-X corporation with the expertise in brain imaging methods and open
source software development at the Henry H. Wheeler Brain Imaging Center at UC
Berkeley, we propose to

\begin{itemize}
\item{Develop new needlet based algorithms for processing diffusion weighted
    MRI data.}
\item{Develop new software libraries implementing these algorithms which take
    advantage of massively parallel GPU architectures.}
\item{Develop a user base for the libraries, which will then be developed into
    a commercial software package.}
\end{itemize}

The result of this collaboration will be to provide the research and clinical
medical communities with unique, state-of-the-art diffusion weighted MRI
software.

\section*{Proposed technical approach}
Many of the algorithms for reconstruction of diffusion weighted MRI data can be
cast as a deconvolution problem for ODF\cite{JIA-VEM-2007}. The ODF, $f\left(\bomega\right)$, gives the volume
fraction of fibers oriented along the direction $\bomega\in\mathbb{S}^2$, where
$\mathbb{S}^2$ is the unit sphere. Mathematically, the problem is to
reconstruct $f$, for each voxel, from the equation
\begin{equation*}
  S\left(\bomega\right) = \int_{\mathbb{S}^2} R\left(\bomega,\bomega'\right)
  f\left(\bomega'\right)d\bomega', 
\end{equation*}  
where $S$ is the measured signal and $R$ is the kernel for the specific imaging
method. We propose to invert this equation using ideas from
\cite{KE-NG-PI-2009} with needlets constructed using new quadratures from
\cite{AHR-BEY-2009} and optimized filters. In a needlet representation, the ODF
will be sparse, that is, since there are only a few $(\lesssim 4)$ fibers per
voxel, only a few needlet coefficients will be non-negligible. Thus, sparse
signal recovery techniques will be used to estimate needlet
coefficients. Because of the localized nature of the needlets and sparsity of
the signal (in the needlet basis), this method will be more robust for
determining fiber direction than those based on spherical harmonics.

The numerical algorithms used to solve the above equation with needlets leads to many $(\sim 10^6)$ small $(\lesssim 256\times256)$ independent linear systems of equations (one system per voxel), which can be efficiently solved on GPUs.\emph{...more on gpu computing -- Paul}

\section*{Commercialization}
Tech-X was founded in Boulder, CO in 1994 by John R. Cary, CEO and Professor of Physics at the University of Colorado and Svetlana Shasharina, Ph.D., Vice-President of Distributed Technologies. With a staff of more than 60 people, of whom 60\% have Ph.D. degrees in physics, mathematics, engineering, and computer science, Tech-X has strong expertise in computational simulations, data analysis, and distributed computing applied to engineering and scientific applications.  The company has been successful throughout its history in developing high-performance software solutions for research in high-energy physics, fusion, particle accelerators, and similar technologies.

\emph{specifics about this project -- Cory and Paul}

\section*{Key personnel}

Dr. Fernando Perez bio

Dr. Matthew Brett bio

Dr. Paul Mullowney bio

Dr. Cory Ahrens received his BS and MS degrees in nuclear engineering from Kansas State University and the University of Michigan in 1999 and 2001, respectively. In 2006, Dr. Ahrens completed a Ph.D. in Applied Mathematics at the University of Colorado, Boulder.


\bibliographystyle{plain}
\bibliography{biblio} 


\end{document}
