\documentclass[10pt]{article}
\usepackage{fancyhdr,amssymb,amsmath,graphicx}
\usepackage{hyperref,fullpage}
\setlength{\topmargin}{-36pt}
\setlength{\headheight}{42pt}
\setlength{\headsep}{10pt}
\setlength{\footskip}{0pt}
\renewcommand{\labelitemi}{{\Large\textperiodcentered}}
\pagenumbering{empty}
\pagestyle{fancy}
% ----------------------------------------

\begin{document}
%\layout
\sloppy
\lhead{{\Huge Tech-X \\ Corporation}}
\rhead{{\large 5621 Arapahoe Ave, Suite A\\
       \large Boulder, CO 80303}}
\chead{\parbox{5cm}{}\includegraphics[width=65pt]{TX_Logo_rev.pdf}}

%Second thicker line
\setlength{\unitlength}{1in}
\begin{picture}(5,0.025)
  \linethickness{0.5mm}
  \put(-0.2,-0.074){\line(1,0){6.5}}
\end{picture}

%Footer infomation
\lfoot{{Phone: 303.448.0727}}
\cfoot{{\url{www.txcorp.com}}}
\rfoot{{Fax: 303.448.7756}}

\newcommand{\bomega}{{\boldsymbol{\omega}}}

%Title
\vspace{5pt}
\begin{center} {\huge{
      New computationally efficient  \\
      \vspace{10pt} algorithms for Diffusion MRI} }
\end{center}

\section*{Summary}

We propose to develop an open source library for the analysis, diagnosis and
visualization of diffusion imaging datasets that includes novel algorithmic
approaches to the representation of the underlying data as well as optimized
components based on the use of specialized fast hardware available today.  This
library will be contributed to existing NIH-funded open source tools for
neuroimaging and will form the basis for the development of an FDA-approved,
commercially supported package for clinical analysis of diffusion imaging data.

Diffusion weighted MRI techniques hold the promise of \emph{in vivo} imaging of
fiber tracts in the brain. This type of imaging is proving to be of enormous
value in understanding the connection structure of the brain, an essential
component for description of both healthy and diseased brain function.

The analysis of diffusion weighted images requires estimating the direction of
motion of water molecules at each location (voxel) in the brain, providing an
Orientation Distribution Function (ODF). Standard approaches represent
the ODF using expansions in spherical harmonics, a set of
functions that constitute the spherical analog of sines and cosines in
classical Fourier analysis.  Spherical harmonics share some key limitations of
sines and cosines, namely their inability to efficiently and accurately
represent localized features in signals. For the study of diffusion data, this
is of particular concern, as many of the key features of the data arise
precisely from complex local patterns of anatomy such as crossing neural
fibers.

Wavelet techniques are a natural mathematical tool for analysis of signals with
localized features, and they have been successfully used in many fields where
the signals can be defined on domains such as a line, plane or cube. However, there
are fundamental difficulties in constructing the analog of wavelets for a spherical
domain, and only recently have these been successfully tackled, with the
introduction of ``needlets'', a wavelet-like construction that respects the
topology of the sphere.

Use of graphical processing units (GPUs) in scientific computing has
increased tremendously. GPUs offer massively-parallel processing capabilities
and it is not uncommon to achieve a factor of 50 in speed up over CPU-based
codes. Because of the independence of voxels and the amount of data in
diffusion MRI, use of GPUs to process diffusion MRI data will offer significant
performance improvements.

By combining the expertise in computational harmonic analysis and GPU computing
at Tech-X corporation with the expertise in brain imaging methods and open
source software development at the Henry H. Wheeler Brain Imaging Center at UC
Berkeley, we propose to
\begin{itemize}
\item{Develop new needlet based algorithms for processing diffusion weighted
    MRI data.}
\item{Develop new software libraries implementing these algorithms which take
    advantage of massively parallel GPU architectures.}
\item{Develop a user base for the libraries, which will then be developed into
    a commercial software package.}
\end{itemize}
The result of this collaboration will be to provide the research and clinical
medical communities with unique, state-of-the-art diffusion weighted MRI
software.

\section*{Proposed technical approach}
The reconstruction of diffusion weighted MRI data can be cast as a
deconvolution [\cite{Lenglet2008, JIA-VEM-2007}]; mathematically, the problem is
to reconstruct the probability of diffusion $p({\mathbf R}|\tau)$, for each
voxel, from the equation
\begin{equation*}
  S\left({\mathbf q},\tau\right) = \int_{\mathbb{R}^3} p({\mathbf R}|\tau)
e^{-2\pi i {\mathbf q}\cdot {\mathbf R}} d{\mathbf R}
\end{equation*}
where $S$ is the signal attenuation measured when a field gradient along
direction ${\mathbf q}$ acts over a time interval $\tau$.  The full probability
$p$ can be approximated by sampling with field gradients over the surface of a
sphere at fixed gradient strength, which provides the Orientation Distribution
Function (ODF). We propose to estimate the ODF this equation using ideas from
\cite{KE-NG-PI-2009} with needlets constructed using new quadratures from
\cite{AHR-BEY-2009} and optimized filters.  The ability of needlets to
faithfully represent localized features will provide for more stable
tractography analysis, as the artifacts inherent in global bases
(e.g. spherical harmonics) can lead to spurious fiber identifications and other
problems.  In order to accurately estimate higher order coefficients of the
needlet expansion, the ODF reconstruction is cast as a sparse reconstruction
problem, for which the numerical algorithms used lead to many $(\sim 10^6)$
small $(\lesssim 256\times256)$ independent linear systems of equations (one
system per voxel), which can be efficiently solved on GPUs. These linear systems are
solved using minimization techniques, for example, bi-conjugate gradients. Tech-X currently
has libraries of GPU-accelerated numerical solvers for tackling these problems
Generalizing these codes to handle a wide range of problems will be an important
component of this project.

\section*{Commercialization}
Tech-X was founded in Boulder, CO in 1994. With a staff of more than 60 people,
of whom 60\% have Ph.D. degrees in physics, mathematics, engineering, and
computer science, Tech-X has strong expertise in computational simulations, GPU
software development, data analysis, and distributed computing applied to
engineering and scientific applications.

In the initial phase of this project, we will develop an open source library
that will be incorporated into the Nipy project (http://nipy.sourceforge.net),
an NIH-funded collection of open source tools for multiple tasks in the
analysis of neuroimaging data.  All software at this stage will be contributed
to Nipy under the terms of the BSD Software License, which allows for further
commercial development. By developing open source software, the broader scientific
community will be to analyze and test our new methodologies. This will allow for
continuous community critique and development, which ultimately will define the
utility of this approach. Once the methodologies gain traction in the neuroimaging
community, Tech-X will then develop an FDA approved, commerical software package
for clinical use. During phase I, we will explore the current market conditions
for this technology and develop a commercialization plan based on those results.

\section*{Key personnel}
This project brings together of number of individuals with varying expertise to make this project successful. These individuals are:
\begin{itemize}
  \item{Dr. Fernando Perez received his PhD in Theoretical Physics from the University
of Colorado in 2002, where he then worked as a postdoctoral researcher in
Applied Mathematics with professor Gregory Beylkin, on the development of fast
multiresolution algorithms for the solution of partial differential equations.
He has made extensive contributions to the development of modern open source
tools for scientific computing in Python. He is an Associate Researcher at the
Brain Imaging Center of the University of California, Berkeley.}

  \item{Dr. Matthew Brett bio}
  
  \item{Dr. Paul Mullowney received his PhD in Applied Mathematics at the University of Colorado at Boulder. He will contribute his expertise in high-performance computating (GPUs) to this project in order to provide efficient computational solutions for the developed algorithms.}

  \item{Dr. Cory Ahrens received Ph.D. in Applied Mathematics at the University of Colorado at Boulder in 2006. Dr. Ahrens will contribute his expertise in developing wavelet based algorithms for the sphere.}
  
\end{itemize}


\bibliographystyle{plain}
\bibliography{biblio,diffusion_needlets}

\end{document}
